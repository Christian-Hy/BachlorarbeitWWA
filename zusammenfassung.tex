\chapter{Zusammenfassung}
In der Ringtheorie wird das Konzept der Primzahl auf die Elemente eines beliebigen kommutativen unitären Rings verallgemeinert. Die entsprechenden Begriffe sind Primelement und irreduzibles Element.

Die Primzahlen und deren Negative sind dann genau die Primelemente und auch genau die irreduziblen Elemente des Rings der ganzen Zahlen. In faktoriellen Ringen, das sind Ringe mit eindeutiger Primfaktorisierung, fallen die Begriffe Primelement und irreduzibles Element zusammen; im Allgemeinen ist die Menge der Primelemente jedoch nur eine Teilmenge der Menge der irreduziblen Elemente.

Insbesondere im zahlentheoretisch bedeutsamen Fall der Dedekindringe übernehmen Primideale die Rolle der Primzahlen.
