%!TEX root = thesis.tex

\chapter{Einleitung}

Eine Primzahl ist eine natürliche Zahl, die größer als 1 und ausschließlich durch sich selbst und durch 1 teilbar ist. 
Eine Primzahl ist also eine natürliche Zahl, die genau zwei natürliche Zahlen als Teiler hat.
Die kleinsten Primzahlen sind:
\begin{equation*}
      \mathbb{P} = \{2,3,5,7,11,13,17,19,23,29,31,37\dots\}
\end{equation*}

Eine natürliche Zahl größer als 1 heißt prim, wenn sie eine Primzahl ist, andernfalls heißt sie zusammengesetzt.
Die Zahlen 0 und 1 sind weder prim noch zusammengesetzt.
Das Wort „Primzahl“ kommt aus dem Lateinischen (numerus primus) und bedeutet „die erste Zahl“.
Die Bedeutung der Primzahlen $\mathbb{P}$ für viele Bereiche der Mathematik beruht auf drei Folgerungen aus dieser Definition:

\begin{itemize}
    \item Existenz und Eindeutigkeit der Primfaktorzerlegung: Jede natürliche Zahl, die größer als 1 und selbst keine Primzahl ist, lässt sich als Produkt von mindestens zwei Primzahlen schreiben.
        Diese Produktdarstellung ist bis auf die Reihenfolge der Faktoren eindeutig. 
    
    \item Zum Beweis dient das Lemma von Euklid: Ist ein Produkt zweier natürlicher Zahlen durch eine Primzahl teilbar, so ist mindestens einer der Faktoren durch sie teilbar.

    \item Primzahlen lassen sich nicht als Produkt zweier natürlicher Zahlen, die beide größer als 1 sind, darstellen.
\end{itemize}
    
Diese Eigenschaften werden in der Algebra für Verallgemeinerungen des Primzahlbegriffs genutzt.
Schon im antiken Griechenland interessierte man sich für die Primzahlen und entdeckte einige ihrer Eigenschaften. 
Obwohl Primzahlen seit damals stets einen großen Reiz auf die Menschen ausübten, sind viele die Primzahlen betreffenden Fragen bis heute ungeklärt, darunter solche, die mehr als hundert Jahre alt und leicht verständlich formulierbar sind.
Dazu gehören die Goldbachsche Vermutung, wonach außer 2 jede gerade Zahl als Summe zweier Primzahlen darstellbar ist, und die Vermutung, dass es unendlich viele Primzahlzwillinge gibt (das sind Paare von Primzahlen, deren Differenz gleich 2 ist).

Über 2000 Jahre lang konnte man keinen praktischen Nutzen aus dem Wissen über die Primzahlen ziehen. 
Dies änderte sich erst mit dem Aufkommen elektronischer Rechenmaschinen, bei denen die Primzahlen beispielsweise in der Kryptographie eine zentrale Rolle spielen.
