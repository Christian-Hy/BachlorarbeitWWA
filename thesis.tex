\documentclass{scrbook}

%!TEX root = thesis.tex

% Set german to default language and load english as well
\usepackage[english,ngerman]{babel}

% Set UTF8 as input encoding
\usepackage[utf8]{inputenc}

% Set T1 as font encoding
\usepackage[T1]{fontenc}
% Load a slightly more modern font
\usepackage{lmodern}
% Use the symbol collection textcomp, which is needed by listings.
\usepackage{textcomp}
% Load a better font for monospace.
\usepackage{courier}

% Set some options regarding the document layout. See KOMA guide
\KOMAoptions{%
  paper=a4,
  fontsize=12pt,
  parskip=half,
  headings=normal,
  BCOR=1cm,
  headsepline,
  bibliography=openstyle,
  DIV=12}

% do not align bottom of pages
\raggedbottom

% set style of captions
\setcapindent{0pt} % do not indent second line of captions
\setkomafont{caption}{\small}
\setkomafont{captionlabel}{\bfseries}
\setcapwidth[c]{0.9\textwidth}

% set the style of the bibliography
\bibliographystyle{alpha}

% load extended tabulars used in the list of abbreviation
\usepackage{tabularx}

% load the color package and enable colored tables
\usepackage[table]{xcolor}

% define new environment for zebra tables
\newcommand{\mainrowcolors}{\rowcolors{1}{maincolor!25}{maincolor!5}}
\newenvironment{zebratabular}{\mainrowcolors\begin{tabular}}{\end{tabular}}
\newcommand{\setrownumber}[1]{\global\rownum#1\relax}
\newcommand{\headerrow}{\rowcolor{maincolor!50}\setrownumber1}

% add main color to section headers
\addtokomafont{chapter}{\color{maincolor}}
\addtokomafont{section}{\color{maincolor}}
\addtokomafont{subsection}{\color{maincolor}}
\addtokomafont{subsubsection}{\color{maincolor}}
\addtokomafont{paragraph}{\color{maincolor}}

% do not print numbers next to each formula
\usepackage{mathtools}
\mathtoolsset{showonlyrefs}
% left align formulas 
\makeatletter
\@fleqntrue\let\mathindent\@mathmargin \@mathmargin=\leftmargini
\makeatother

% Allow page breaks in align environments
\allowdisplaybreaks

% header and footer
\usepackage{scrpage2}
\pagestyle{scrheadings}
\setkomafont{pagenumber}{\normalfont\sffamily\color{maincolor}}
\setkomafont{pageheadfoot}{\normalfont\sffamily}
\setheadsepline{0.5pt}[\color{maincolor}]

% load TikZ to draw diagrams
\usepackage{tikz}

% load additional libraries for TikZ
\usetikzlibrary{%
  automata,%
  positioning,%
}

% set some default options for TikZ -- in this case for automata
\tikzset{
  every state/.style={
    draw=maincolor,
    thick,
    fill=maincolor!18,
    minimum size=0pt
  }
}

% load listings package to typeset sourcecode
\usepackage{listings}

% set some options for the listings package
\lstset{%
  numbers=none,%
  showstringspaces=false,%
  upquote=true,%
  basicstyle=\ttfamily,%
  keywordstyle=\color{keywordcolor}\slshape,%
  commentstyle=\color{commentcolor}\itshape,%
  stringstyle=\color{stringcolor},%
  mathescape=true,%
  keepspaces=true}

% load the AMS math library to typeset formulas
\usepackage{amsmath}
\usepackage{amsthm}
\usepackage{thmtools}
\usepackage{amssymb}

% load the paralist library to use compactitem and compactenum environment
\usepackage{paralist}

% load varioref and hyperref to create nicer references using vref
\usepackage[ngerman]{varioref}
\usepackage{hyperref}

% setup hyperref
\hypersetup{breaklinks=true,
            pdfborder={0 0 0},
            ngerman,
            pdfhighlight={/N},
            pdfdisplaydoctitle=true}

% define german names for referenced elements
% (vref automatically inserts these names in front of the references)
\labelformat{figure}{Abbildung\ #1}
\labelformat{table}{Tabelle\ #1}
\labelformat{appendix}{Anhang\ #1}
\labelformat{chapter}{Kapitel\ #1}
\labelformat{section}{Abschnitt\ #1}
\labelformat{subsection}{Unterabschnitt\ #1}
\labelformat{subsubsection}{Unterunterabschnitt\ #1}

% define theorem environments
\declaretheorem[numberwithin=chapter,style=plain]{Theorem}
\labelformat{Theorem}{Theorem\ #1}

\declaretheorem[sibling=Theorem,style=plain]{Lemma}
\labelformat{Lemma}{Lemma\ #1}

\declaretheorem[sibling=Theorem,style=definition]{Definition}
\labelformat{Definition}{Definition\ #1}

\declaretheorem[sibling=Theorem,style=definition]{Beispiel}
\labelformat{Beispiel}{Beispiel\ #1}

\declaretheorem[sibling=Theorem,style=definition]{Bemerkung}
\labelformat{Bemerkung}{Bemerkung\ #1}

% Use this file to define some macros you need in your thesis. A macro is a short command that inserts some mathematical symbols or texts you do not want to retype each time you need some. I recommend to use as many macros as possible, because you are able to change them later. For example if you use the same macro each time you need to give the formal semantics of an expression you can easily change the appearance of these brackets by updating the macro later on.

% Set of natural numbers
\newcommand{\N}{\mathbb{N}}

% The default epsilon does not look very nice
\let\epsilon\varepsilon

% If you need to use mathematical expressins like an epsilon in the section titles of your thesis you will end up with warnings that these special symbols cannot be included in the PDF favorites. The following macro uses the mathematical symbol during the text of the thesis and the string "Epsilon" in the PDF favorites.
\newcommand{\pdfepsilon}{\texorpdfstring{$\epsilon$}{Epsilon}}


% Set title and author used in the PDF meta data
\hypersetup{
  pdftitle={Wie schreibe ich eine Masterarbeit?},
  pdfauthor={Erika Mustermann}
}

% Depending on which of the following two color schemes you import your thesis will be in color or grayscale. I recommend to generate a colored version as a PDF and a grayscale version for printing.

%!TEX root = thesis.tex

% define color of example university
\xdefinecolor{exampleuniversity}{rgb}{1, 0.5, 0}

\colorlet{maincolor}{ForestGreen}

\colorlet{stringcolor}{green!60!black}
\colorlet{commentcolor}{black!50}
\colorlet{keywordcolor}{maincolor!80!black}

\newcommand{\imagesuffix}{-color}

%\colorlet{maincolor}{black}

\colorlet{stringcolor}{black}
\colorlet{commentcolor}{black!50}
\colorlet{keywordcolor}{black}

\newcommand{\imagesuffix}{-gray}

\newcommand{\duedate}{15. Juli 2016}

\begin{document}
  \frontmatter
  %!TEX root = thesis.tex

\begin{titlepage}
  \thispagestyle{empty}

  \vskip1cm

  \pgfimage[height=2.5cm]{uni-logo-example\imagesuffix}
  
  \vskip2.5cm
  
  \LARGE
  
  \textbf{\sffamily\color{maincolor}Spaß mit Primzahlen}

  \textit{Primzahlen und Primfaktorzerlegung usw. usf}

  \normalfont\normalsize

  \vskip2em
  
  \textbf{\sffamily\color{maincolor}Masterarbeit}

  im Rahmen des Studiengangs \\
  \textbf{\sffamily\color{maincolor}Informatik} \\
  der Universität zu Lübeck

  \vskip1em

  vorgelegt von \\
  \textbf{\sffamily\color{maincolor}Fabian Grieser und Christian Hyttrek}

  \vskip1em
  
  ausgegeben und betreut von \\
  \textbf{\sffamily\color{maincolor}Prof. Dr. Erika Musterfrau}

  \vskip1em

  mit Unterstützung von\\
  Lieschen Müller

  \vskip1em

  Die Arbeit ist im Rahmen einer Tätigkeit bei der Firma Muster GmbH entstanden.


  \vfill

  Musterhausen, den \duedate
\end{titlepage}

  %!TEX root = thesis.tex

\cleardoublepage
\thispagestyle{plain}
\vspace*{\fill}

\section*{Erklärung}

Hiermit erkläre ich an Eides statt, dass ich die vorliegende
Arbeit ohne unzulässige Hilfe Dritter und ohne die Benutzung anderer
als der angegebenen Hilfsmittel selbständig verfasst habe;
die aus anderen Quellen direkt oder indirekt übernommenen Daten und Konzepte
sind unter Angabe des Literaturzitats gekennzeichnet.

\vskip2cm

\rule{5cm}{0.4pt}\\
(Max Mustermann)\\
Musterhausen, den \duedate

  %!TEX root = thesis.tex

\cleardoublepage
\thispagestyle{plain}

\pdfbookmark{Kurzfassung}{kurzfassung}
\paragraph{Kurzfassung}
Die Ihnen vorliegende Arbeit befasst sich mit dem hochinteressanten Themengebiet der Primzahlen. 
Eine Primzahl ist bekanntlich eine natürliche Zahl, die größer ist als 1 und nur durch sich selbst und 1 teilbar ist.
In dieser Arbeit werden Sie lesen was die Primfaktorzerlegung ist und irgendwelche Anwendungszwecke.

\cleardoublepage
\thispagestyle{plain}

\foreignlanguage{english}{%
\pdfbookmark{Abstract}{abstract}
\paragraph{Abstract} 
   The work in hand is about the marvellous topic of prime numbers. 
   A prime number is, as we all know, a natural number, which is greater than 1 and only divisable by 1 and the number itself.
   In this work you will read how the prime factorization is done and some use cases.
}


  \cleardoublepage
  \phantomsection
  \pdfbookmark{Inhaltsverzeichnis}{tableofcontents}
  \markboth{Inhaltsverzeichnis}{}
  \tableofcontents

  \mainmatter
  \chapter{Einleitung}

Die Einleitung führt zum eigentlichen Thema dieser Arbeit hin. Dabei wird ein großer Bogen gespannt, in dem die Relevanz und der Kontext der untersuchten Thematik deutlich wird. Grundlegende Begriffe aus dem Titel und der Kurzfassung sollten aufgegriffen und definiert werden. Unterstützend können Zitate herangezogen werden, die der Arbeit einen Rahmen geben.

\section{Verwandte Arbeiten}

Eine wichtiger Abschnitt der Einleitung stellt einen Überblick über verwandte Arbeiten dar. Was wurde bereits in der Literatur untersucht und ist \emph{nicht} Thema dieser Arbeit?

\section{Aufbau der Arbeit}

Neben dieser Einleitung und der Zusammenfassung am Ende gliedert sich diese Arbeit in die folgenden drei Kapitel.
\begin{description}
  \item[\ref{chapter-basics}] beschreibt die für diese Arbeit benötigten Grundlagen. In diesem Kapitel werden \ldots, \ldots und \ldots eingeführt, da diese für die folgenden Kapitel dringend benötigt werden.
  \item[\ref{chapter-konzept}] stellt das eigentliche Konzept vor. Dabei handelt es sich um ein Konzept zur Verbesserung der Welt. Das Kapitel gliedert sich daher in einen globalen und einen lokalen Ansatz, wie die Welt zum Besseren beeinflusst werden kann.
  \item[\ref{chapter-evaluation}] beinhaltet eine Evaluation des Konzeptes aus dem vorherigen Kapitel. Anhand von Simulationen wird in diesem Kapitel untersucht, wie die Welt durch konkrete Maßnahmen deutlich verbessert werden kann.
\end{description}


  %!TEX root = thesis.tex

\chapter{Grundlagen}
\label{chapter-basics}

Dieses Kapitel beschreibt alle für die Arbeit notwendigen Grundlagen.

\section{Zielgruppe}

Die Zielgruppe einer Abschlussarbeit sind natürlich in erster Linie die Gutachter, die am Ende die Arbeit lesen und bewerten. Als Richtlinie, welches Wissen beim Leser einer solchen Arbeit vorausgesetzt werden kann, sollte man sich allerdings einen Kommilitonen des gleichen Studiengangs vorstellen, der in einem anderen Fachbereich seine Abschlussarbeit schreibt. Für diesen sollten wenigstens alle wesentlichen Definitionen enthalten sein.

\section{Umfang}

Eine Abschlussarbeit ist allerdings kein Lehrbuch. Entsprechend ist vor allem die Korrektheit von Definitionen wichtig. Zu umfangreiche Beispiele für in der Literatur bereits zur genüge untersuchte Grundlagen sollten vermieden werden, um den Leser nicht zu langweilen.

\section{Quellen}

Eine wesentliches Charakteristikum von Grundlagen ist, das diese nicht vom Autor der Arbeit erfunden wurde. Entsprechend ist gerade in den Grundlagen darauf zu achten, ausreichend Quellen anzugeben. Eine gute Regel ist dabei, immer den Erfinder bzw. das erste Auftauchen eines Konzeptes in der Literatur und mindestens ein gut verständliche neuere Quelle anzugeben. In den meisten Fällen hat sich seit der Einführung einer neuen Idee einiges getan, so das neuere Quellen meistens einen besseren Einstieg in das Thema bieten. Es gehört sich aber, den Erfinder immer mit zu zitieren, da dieser die initiale Idee hatte.


  %!TEX root = thesis.tex

\chapter{Konzept}
\label{chapter-konzept}

In diesem Kapitel wird die eigentliche Erkenntnis dieser Arbeit beschrieben. Der Aufbau dieses Kapitels hängt stark vom Thema der Arbeit ab. Die in dieser Vorlage vorgeschlagenen Kapitel sind auch nur als Vorschlag und auf keinen Fall als verbindlich zu verstehen.

Die folgenden Abschnitte dieses Kapitels enthalten Beispiele für die diversen Inhaltselemente einer Arbeit.

\section{Quellen}

Quellen sind wichtig für gutes wissenschaftliches Arbeiten. Eine Quelle kann dabei zum Beispiel
\begin{compactitem}
  \item ein Beitrag in einer Zeitschrift \cite{MopOverview},
  \item ein Beitrag in einem Sammlungsband \cite{moore},
  \item ein Buch \cite{scala},
  \item ein Beitrag im Berichtsband einer Konferenz \cite{rltl},
  \item ein technischer Bericht \cite{bitkom},
  \item eine Abschlussarbeit \cite{RltlConv},
  \item ein noch nicht veröffentlichter Artikel \cite{ptLTL} oder
  \item ein Artikel auf einer Website \cite{codecommit} sein.
\end{compactitem}

Dabei ist zu beachten, dass nicht veröffentlichte Artikel und insbesondere Webseiten nur in Ausnahmefällen gute Quellen sind, da diese nicht durch Fachleute begutachtet wurden.

Im Bereich der Informatik können Quellenangaben im Bib\TeX-Format direkt der dblp\footnote{zum Beispiel \url{http://dblp.uni-trier.de}} entnommen werden.

\section{Tabellen}

In \vref{tbl-prozessoren} sehen wir ein Beispiel für eine Tabelle.

\begin{table}
  \centering
  \begin{zebratabular}{llr}
    \headerrow Jahr & Prozessor & MHz \\
    1975 & 6502 (C64) & 1 \\
    1985 & 80386 & 16 \\
    2005 & Pentium 4 & 2\,800 \\
    2030 & Phoenix 3 & 7\,320\,000 \\
    \hiderowcolors
    2050 & \ldots \\
    2070 & \ldots
  \end{zebratabular}
  \caption[Rechengeschwindigkeit von Computern]{Rechengeschwindigkeit von Computern. Inhaltlich vollkommen egal, ist dies doch ein sehr schönes Beispiel für eine Tabelle.}
  \label{tbl-prozessoren}
\end{table}

\section{Abbildungen und Diagramme}

In \vref{fig-flower} sehen wir ein Beispiel für eine Abbildung, die aus einer externen Grafik geladen wurde. In \vref{fig-buechi} sehen wir ein Beispiel für eine Abbildung, die in \LaTeX\ generiert wurde.

\begin{figure}
  \centering
  \pgfimage[width=.5\textwidth]{flower}
  \caption[Kurzfassung der Beschreibung für das Abbildungsverzeichnis]{Lange Version der Beschreibung, die direkt unter der Abbildung gesetzt wird. Es ist wichtig, für jede Abbildung eine umfangreiche Beschreibung anzugeben, da der Leser beim ersten Durchblättern der Arbeit vor allem an den Abbildungen hängen bleibt.}
  \label{fig-flower}
\end{figure}

\begin{figure}
  \centering
  \begin{tikzpicture}[
      node distance=15ex,
      auto,
      on grid,
      shorten >=1pt
    ]
    \node [state, initial] (q0) {$q_0$};
    \node [state, accepting, right=of q0] (q1) {$q_1$};
    \path[->]
      (q0) edge node {$a$} (q1);
  \end{tikzpicture}
  \caption[Graph des Büchi-Automaten $\hat A$.]{Graph des Büchi-Automaten $\hat A$. Der Zustand $q_1$ hat dabei keine ausgehende Kante. Der Zustand ist trotzdem akzeptierend, da beide enthaltenen Zustände von $\acute A$ akzeptierend sind. Die naive Anwendung des Leerheitstests auf alternierenden Büchi-Automaten liefert in diesem Fall also zu viele akzeptierende Zustände.}
  \label{fig-buechi}
\end{figure}

\section{Quelltext}

Quelltext sollte in Abschlussarbeiten nur äußerst sparsam eingesetzt werden. Wichtig ist insbesondere, dass Quelltextauszüge sorgsam ausgewählt und gut erklärt werden.

\begin{lstlisting}[language=Java,gobble=2]
  public class Main {
    // Hello Word in Java
    public static void main(String[] args) {
      System.out.println("Hello World");
    }
  }
\end{lstlisting}

\section{Pseudocode}

Um Algorithmen zu erklären ist Pseudocode viel besser geeignet als Quelltext. Im Pseudocode kann man alles unwichtige weglassen und sich auf die mathematische Modellierung des Algorithmus konzentrieren. So kann die Struktur des Verfahrens unabhängig von Implementierungsdetails des jeweiligen Frameworks erklärt werden.

\begin{lstlisting}[style=pseudo,gobble=2]
  // Schleife von 1 bis 5
    for $i \gets 1$ to $5$ do
      while $S[i] \neq S[S[i]]$ do
        $S[i] \gets S[S[i]]$
\end{lstlisting}

\section{Formeln mit \pdfepsilon}

Das $\epsilon$ in der Überschrift dieses Abschnitts ist ein Beispiel für ein mathematisches Symbol, dass in den PDF-Lesezeichen als reiner Text gesetzt wird. Siehe \texttt{macros.tex}. In dieser Datei wird auch $n \in \N$ definiert.

Wir wissen aus der Analysis, dass
\begin{align}
  f(x) &= x^2 + px + q
\end{align}
Nullstellen bei
\begin{align}
  x_1 &= -\frac p2 + \sqrt{\frac{p^2}4 - q} \text{ und}\\
  x_2 &= -\frac p2 - \sqrt{\frac{p^2}4 - q}
\end{align}
hat.

\section{Theoreme}

\begin{Theorem}[Wichtiger Satz]
  Ein wichtiger Satz.
\end{Theorem}

\begin{proof}
  Der Beweis.
\end{proof}

\begin{Lemma}[Unwichtiger Hilfssatz]
  Der Satz \ldots
\end{Lemma}

\begin{proof}
  \ldots und sein Beweis.
\end{proof}

\begin{Definition}[Definition]
  Eine \emph{Definition} ist die Bestimmung eines Begriffs oder eines mathematischen Zusammenhangs.
\end{Definition}

\begin{Beispiel}
  Ein Beispiel.
\end{Beispiel}

\begin{Bemerkung}
  Beispiele und Bemerkungen werden nicht in das Verzeichnis der Theoreme und Definitionen aufgenommen.
\end{Bemerkung}


  \chapter{Evaluierung}
\label{chapter-evaluation}

In der Evaluierung wird das Ergebnis dieser Arbeit bewertet. Eine praktische Evaluation eines neuen Algorithmus kann zum Beispiel durch eine Implementierung geschehen. Je nach Thema der Arbeit kann sich natürlich auch die gesamte Arbeit eher im praktischen Bereich mit einer Implementierung beschäftigen. In diesem Fall gilt es am Ende der Arbeit insbesondere die Implementierung selber zu evaluieren. Wesentliche Fragen dabei können sein:
\begin{compactitem}[--]
  \item Was funktioniert jetzt besser als vor meiner Arbeit?
  \item Wie kann das praktisch eingesetzt werden?
  \item Was sagen potenzielle Anwender zu meiner Lösung?
\end{compactitem}

\section{Implementierungen}

Wenn Implementierungen umfangreich beschrieben werden, ist darauf zu achten, den richtigen Mittelweg zwischen einer zu detaillierten und zu oberflächlichen Beschreibung zu finden. Eine Beschreibung aller Details der Implementierung ist in der Regel zu detailliert, da die primäre Zielgruppe einer Abschlussarbeit sich nicht im Detail in den geschriebenen Quelltext einarbeiten will. Die Beschreibung sollte aber durchaus alle wesentlichen Konzepte der Implementierung enthalten. Gerade bei einer Abschlussarbeit am Institut für Softwaretechnik und Programmiersprachen lohnt es sich, auf die eingesetzten Techniken und Programmiersprachen einzugehen. Ich würde in einer solchen Beschreibung auch einige unterstützende Diagramme erwarten.


  \chapter{Zusammenfassung und Ausblick}
\label{chapter-fazit}

Die Zusammenfassung greift die in der Einleitung angerissenen Bereiche wieder auf und erläutert, zu welchen Ergebnissen diese Arbeit kommt. Dabei wird insbesondere auf die neuen Erkenntnisse und den Nutzen der Arbeit eingegangen.

Im anschließenden Ausblick werden mögliche nächste Schritte aufgezählt, um die Forschung an diesem Thema weiter voranzubringen. Hier darf man sich nicht scheuen, klar zu benennen, was im Rahmen dieser Arbeit nicht bearbeitet werden konnte und wo noch weitere Arbeit notwendig ist.

  \appendix

  %!TEX root = thesis.tex

\chapter{Anhang}

Dieser Anhang enthält tiefergehende Informationen, die nicht zur eigentlichen Arbeit gehören.

\section{Abschnitt des Anhangs}

In den meisten Fällen wird kein Anhang benötigt, da sich selten Informationen ansammeln, die nicht zum eigentlichen Inhalt der Arbeit gehören. Vollständige Quelltextlisting haben in ausgedruckter Form keinen Wort und gehören daher weder in die Arbeit noch in den Anhang. Darüber hinaus gehören Abbildungen bzw. Diagramme, auf die im Text der Arbeit verwiesen wird, auf keinen Fall in den Anhang.

  \backmatter

  \cleardoublepage
  \phantomsection
  \pdfbookmark{Abbildungsverzeichnis}{listoffigures}
  \listoffigures

  \cleardoublepage
  \phantomsection
  \pdfbookmark{Tabellenverzeichnis}{listoftables}
  \listoftables

  \cleardoublepage
  \phantomsection
  \pdfbookmark{Definitions- und Theoremverzeichnis}{listoftheorems}
  \renewcommand{\listtheoremname}{Definitions- und Theoremverzeichnis}
  \listoftheorems[ignoreall,show={Lemma,Theorem,Korollar,Definition}]

  %!TEX root = thesis.tex

\cleardoublepage
\phantomsection
\pdfbookmark{Abkürzungsverzeichnis}{abbreviations}
\chapter*{Abkürzungsverzeichnis}
\label{section-abbrevs}

\begin{tabularx}{\textwidth}{lX}
  ABA & alternierender Büchi-Automat, engl. \emph{a}lternating \emph{B}üchi \emph{a}utomaton\\
  AFA & alternierender endlicher Automat, engl. \emph{a}lternating \emph{f}inite \emph{a}utomaton\\
  BA & Büchi-Automat, engl. \emph{B}üchi \emph{a}utomaton\\
  BNF & Normalform kontextfreier Grammatiken, engl. \emph{B}ackus--\emph{N}aur \emph{f}orm\\
  DFA & endlicher Automat, engl. \emph{d}eterministic \emph{f}inite \emph{a}utomaton
\end{tabularx}


  \cleardoublepage
  \phantomsection
  \pdfbookmark{Literaturverzeichnis}{bibliography}
  \bibliography{literature}
\end{document}
